\section{Вывод}
Результаты экспериментов соответствуют теоритическим оценкам сложности алгоритмов. Алгоритм Грэхема, использующий АВЛ-дерево, выполняется дольше, чем использующий быструю сортировку, в частности из-за необходимости выделять дополнительную память для вершин, в то время как быстрая сортировка выполняется \textquotedblleft{}на месте\textquotedblright{}, без привлечения дополнительной памяти. 

Результаты проведенных экспериментов в режимах ($A$) и ($B$) отличаются незначительно, что говорит о том, что время работы алгоритма гораздо сильнее зависит от размера входных данных, чем от распределения точек.
Об этом же свидетельствуют результаты эксперимента 2: время работы алгоритма не меняется в зависимости от размера квадрата, в который вписаны точки из входного множества.

На практике реализация алгоритм Грэхема использующая быструю сортировку выполняется быстрее, однако существуют такие наборы входных данных, на которых время работы алгоритма составляет $O(n^2)$. Вероятность получения таких входных данных мала, но не равна нулю. В отличие от этого, алгоритм Грэхема основанный на АВЛ-дереве гарантированно выполняется за $O(n\lg n)$ за счёт свойства сбалансированности АВЛ-дерева.

Таким образом выбор алгоритма зависит от потребностей: обычно алгоритм основанный на быстрой сортировке работает быстрее, однако если требуется гарантированное быстрое время работы на всех возможных входных данных, тогда алгоритм основанный на АВЛ-дереве является более предпочтительным.

\newpage
